\documentclass{article}
\usepackage{geometry}
\usepackage{amsfonts}

\geometry{
a4paper, total={170mm, 257mm},left=20mm,
top=20mm,
}

\begin{document}

\begin{center}
  \bfseries\large
Question Bank

\normalsize

  Subject: Statistics First Paper (Creative)

%  \normalfont\normalsize
 % 11.45a.m.~--~1.45p.m.
\end{center}

\begin{center}
\textbf{Group A}
\end{center}

\begin{enumerate}
  \item
  \textbf{Income and expenditure (both in thousands) of some individuals are collected:}
 

\begin{table}[h]
 \begin{center}
\begin{tabular}{l|l|l|l|l}

Income (x)  & 20 & 30 & 25 & 10 \\ \hline
Expenditure (y) & 15  & 27  & 18 & 5 \\ 
\end{tabular}
\end{center}
\end{table}


  \begin{enumerate}
    \item
	What is a discrete variable? \hfill 1
    \item
    	Can fractional numbers be discrete? Explain briefly.  \hfill 2
    \item
    	Are, in the stem, $\displaystyle \sum_{i=1}^{n} \sum_{i=1}^{n} x_iy_j = \sum_{i=1}^{n} x_iy_i?$ Show statistically. \hfill 3
     \item
     	Prove empirically that sum of square is unequal to square of sum of numbers. \hfill 4
  \end{enumerate}
  
    \item
  \textbf{In the test examination, marks of 11 students in statistics are \\ 90, 92, 93, 49, 44, 88, 80, 58, 83, 71, 76.}
  \begin{enumerate}
    \item
	What is central tendency? \hfill 1
    \item
	When is median better than arithmetic mean? Explain with an example. \hfill 2
    \item  
	Find the 3rd the quartile and 61st percentile from the data and explain.  \hfill 3
    \item
	Do quantiles depend on change of origin and scale. Prove using two examples.\hfill 4
\end{enumerate}


\begin{center}
\textbf{Group B}
\end{center}

 \item
	  \textbf{US Dollar exchange (to taka) in Bangladesh since 1980 to 2005 (after each 5 years) were: \\ 16, 31, 36, 40, 52, 64} 
  
  \begin{enumerate}
    \item
	What are moments? \hfill 1
    \item
	Which moment is equal to the variance? Show mathematically. \hfill 2
    \item  
	Find, from the stem, the first and second raw moments about 1. \hfill 3
    \item
	Find skewness and kurtosis of and explain. \hfill 4
\end{enumerate}

 \item
	  \textbf{GDP (in bn. US\$ PPP) of Bangladesh since 1980 to 1985 according to an estimate \\ of International  Monetary Fund: 41.2, 47.4, 52.0, 56.5, 61.0, 65.3}
  \begin{enumerate}
    \item
	What is time series data? \hfill 1
    \item
	What are the components of a time series model? \hfill 2
    \item  
	Determine the 3-yearly moving average from the data. \hfill 3
    \item
	Find trend of the data using another method (other than (c)), plot both, and comment \\ which is better. \hfill 4
\end{enumerate}

\end{enumerate}

\end{document}