\documentclass{exam}
%\documentclass[11pt,a4paper]{exam}
\usepackage{amsmath,amsthm,amsfonts,amssymb,dsfont}
\usepackage{ifthen}
\usepackage[legalpaper, total={177.8mm, 290mm}]{geometry}
\usepackage{enumerate}% http://ctan.org/pkg/enumerate
\usepackage{multicol}
\usepackage{hhline}
\usepackage[table]{xcolor}


% Accumulate the answers. Unmodified from Phil Hirschorn's answer
% https://tex.stackexchange.com/questions/15350/showing-solutions-of-the-questions-separately/15353
\newbox\allanswers
\setbox\allanswers=\vbox{}

\newenvironment{answer}
{%
    \global\setbox\allanswers=\vbox\bgroup
    \unvbox\allanswers
}%
{%
    \bigbreak
    \egroup
}

\newcommand{\showallanswers}{\par\unvbox\allanswers}
% End Phil's answer


% Is there a better way?
\newcommand*{\getanswer}[5]{%
    \ifthenelse{\equal{#5}{a}}
    {\begin{answer}\thequestion. (a)~#1\end{answer}}
    {\ifthenelse{\equal{#5}{b}}
        {\begin{answer}\thequestion. (b)~#2\end{answer}}
        {\ifthenelse{\equal{#5}{c}}
            {\begin{answer}\thequestion. (c)~#3\end{answer}}
            {\ifthenelse{\equal{#5}{d}}
                {\begin{answer}\thequestion. (d)~#4\end{answer}}
                {\begin{answer}\textbf{\thequestion. (#5)~Invalid answer choice.}\end{answer}}}}}
}

\setlength\parindent{0pt}
%usage \choice{ }{ }{ }{ }
%(A)(B)(C)(D)
\newcommand{\fourch}[5]{
    \par
    \begin{tabular}{*{4}{@{}p{0.23\textwidth}}}
        (a)~#1 & (b)~#2 & (c)~#3 & (d)~#4
    \end{tabular}
    \getanswer{#1}{#2}{#3}{#4}{#5}
}

%(A)(B)
%(C)(D)
\newcommand{\twoch}[5]{
    \par
    \begin{tabular}{*{2}{@{}p{0.46\textwidth}}}
        (a)~#1 & (b)~#2
    \end{tabular}
    \par
    \begin{tabular}{*{2}{@{}p{0.46\textwidth}}}
        (c)~#3 & (d)~#4
    \end{tabular}
    \getanswer{#1}{#2}{#3}{#4}{#5}
}

%(A)
%(B)
%(C)
%(D)
\newcommand{\onech}[5]{
    \par
    (a)~#1 \par (b)~#2 \par (c)~#3 \par (d)~#4
    \getanswer{#1}{#2}{#3}{#4}{#5}
}

\newlength\widthcha
\newlength\widthchb
\newlength\widthchc
\newlength\widthchd
\newlength\widthch
\newlength\tabmaxwidth

\setlength\tabmaxwidth{0.96\textwidth}
\newlength\fourthtabwidth
\setlength\fourthtabwidth{0.25\textwidth}
\newlength\halftabwidth
\setlength\halftabwidth{0.5\textwidth}

\newcommand{\choice}[5]{%
\settowidth\widthcha{AM.#1}\setlength{\widthch}{\widthcha}%
\settowidth\widthchb{BM.#2}%
\ifdim\widthch<\widthchb\relax\setlength{\widthch}{\widthchb}\fi%
    \settowidth\widthchb{CM.#3}%
\ifdim\widthch<\widthchb\relax\setlength{\widthch}{\widthchb}\fi%
    \settowidth\widthchb{DM.#4}%
\ifdim\widthch<\widthchb\relax\setlength{\widthch}{\widthchb}\fi%

% These if statements were bypassing the \onech option.
% \ifdim\widthch<\fourthtabwidth
%     \fourch{#1}{#2}{#3}{#4}{#5}
% \else\ifdim\widthch<\halftabwidth
% \ifdim\widthch>\fourthtabwidth
%     \twoch{#1}{#2}{#3}{#4}{#5}
% \else
%      \onech{#1}{#2}{#3}{#4}{#5}
%  \fi\fi\fi}

% Allows for the \onech option.
\ifdim\widthch>\halftabwidth
    \onech{#1}{#2}{#3}{#4}{#5}
\else\ifdim\widthch<\halftabwidth
\ifdim\widthch>\fourthtabwidth
    \twoch{#1}{#2}{#3}{#4}{#5}
\else
    \fourch{#1}{#2}{#3}{#4}{#5}
\fi\fi\fi}


\begin{document}

\begin{center}
  \bfseries\large
  Sylhet Cadet College

\normalsize
2nd Term-End Examination - 2024

  Class: VII
  
  Subject: Mathematics

  Time: 25 minutes \qquad \qquad  \qquad \qquad \qquad  \qquad \qquad  \qquad Full Marks: 30

%  \normalfont\normalsize
 % 11.45a.m.~--~1.45p.m.
\end{center}


 \begin{center} \textbf{[N.B-Answer all the questions. Each question is worth 01 (one) mark.]}\\


\vspace{0.5cm}

Cadet No. \noindent\rule{1cm}{0.4pt} \qquad Name: \noindent\rule{2cm}{0.4pt}  
Form: \noindent\rule{1cm}{0.4pt} Sign of Invigilator: \noindent\rule{2cm}{0.4pt}
 
\end{center}

\textbf{Short Questions}

\begin{enumerate}

\item If a piece of paper is folded 10 times, how many layers are created? \noindent\rule{2cm}{0.4pt}

\item What is the value of $\frac{2^5}{4^2}$ \noindent\rule{2cm}{0.4pt}

\item What is GCD of $\frac12$ and $\frac14$? \noindent\rule{2cm}{0.4pt}

\item What is Unit Ratio? \noindent\rule{2cm}{0.4pt}

\item In a polygon, what is the relationship between the no. of vertices and
the no. of sides? \noindent\rule{2cm}{0.4pt}

\item Which ratio gives us the value of Pi ($\pi$)? \noindent\rule{2cm}{0.4pt}

\item If $4x+6 = 18, x = ?$  \noindent\rule{2cm}{0.4pt}

\item If $\frac ac = \frac db, b = ?$ \noindent\rule{2cm}{0.4pt}

\item What is the LCM of $\frac 12$ and $\frac 16$ \noindent\rule{2cm}{0.4pt}

\item The ratio of two numbers is 2:3. If the first number is 20, what is 
the other number? \noindent\rule{2cm}{0.4pt}

\item Where do you have to put your finger to hold a disk on it? \noindent\rule{2cm}{0.4pt}

\item $x - 42 = 915; x = ?$ \noindent\rule{2cm}{0.4pt}

\item Write ten thousand in exponent form. \noindent\rule{2cm}{0.4pt}

\item Write 3 mutliples of 0.3, excluding itself. \noindent\rule{2cm}{0.4pt}

\item Write 2 common factors of $\frac12$ and $\frac14s$ \noindent\rule{2cm}{0.4pt}

\item A:B = 5:6 and B:C = 8:12. What is A:B:C? \noindent\rule{2cm}{0.4pt}

\item Two numbers are 10 and 35. Express them as a ratio. \noindent\rule{2cm}{0.4pt}

\item The are of a circle is 28.26. What is its radius? [assume $\pi = 3.14]$ \noindent\rule{2cm}{0.4pt}

\item Name the type of the equation: $ax^2 + bx + c$ \noindent\rule{2cm}{0.4pt}

\item $d \div 20 = 20.$ What is the value of d? \noindent\rule{2cm}{0.4pt}

\item $(2^{-2})^2=?$ \noindent\rule{2cm}{0.4pt}

\item {If $8^{27} \times 8 ^ x = 8 ^{30}, $ what is the value of x? \noindent\rule{2cm}{0.4pt}

\item Write 10 Crore in exponent form. \noindent\rule{2cm}{0.4pt}

\item Price of 4 apples is equal to that of 2 apples and 1 orange. If price of an apples is 5 taka, what is the price of an orange? \noindent\rule{2cm}{0.4pt}

\item A pattern is: $X_n = 3 \times 2^{n-1}$; where $X_n$ is the 
nth value. what is 5th number? \noindent\rule{2cm}{0.4pt}

\item What is the value if 2 is divided by 0.5? \noindent\rule{2cm}{0.4pt}

\item Write down 3 multiples of $\frac 16$ \noindent\rule{2cm}{0.4pt}

\item \textbf{$\frac 13, \frac 14, \frac 24, \frac 49 \rightarrow$} Which is the greatest? \noindent\rule{2cm}{0.4pt}

\item $0.25, 0.125, 0.30, 0.55 \rightarrow$ Which one is a factor of $0.5$ ?  \noindent\rule{2cm}{0.4pt}

\item The radius of a circle is 1 m. What is its circumference? \noindent\rule{2cm}{0.4pt}

\end{enumerate}


 \vspace{2.5cm}

\begin{center}
Quote
\end{center}

\pagebreak
%\newpage  %Uncomment to put on new age
\bigskip

\begin{multicols}{3}
[
Answer Key
]
\showallanswers % Phil Hirschorn
\end{multicols}


\end{document}