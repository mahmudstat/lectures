\documentclass{exam}
%\documentclass[11pt,a4paper]{exam}
\usepackage{amsmath,amsthm,amsfonts,amssymb,dsfont}
\usepackage{ifthen}
\usepackage{enumerate}% http://ctan.org/pkg/enumerate
\usepackage{multicol}



% Accumulate the answers. Unmodified from Phil Hirschorn's answer
% https://tex.stackexchange.com/questions/15350/showing-solutions-of-the-questions-separately/15353
\newbox\allanswers
\setbox\allanswers=\vbox{}

\newenvironment{answer}
{%
    \global\setbox\allanswers=\vbox\bgroup
    \unvbox\allanswers
}%
{%
    \bigbreak
    \egroup
}

\newcommand{\showallanswers}{\par\unvbox\allanswers}
% End Phil's answer


% Is there a better way?
\newcommand*{\getanswer}[5]{%
    \ifthenelse{\equal{#5}{a}}
    {\begin{answer}\thequestion. (a)~#1\end{answer}}
    {\ifthenelse{\equal{#5}{b}}
        {\begin{answer}\thequestion. (b)~#2\end{answer}}
        {\ifthenelse{\equal{#5}{c}}
            {\begin{answer}\thequestion. (c)~#3\end{answer}}
            {\ifthenelse{\equal{#5}{d}}
                {\begin{answer}\thequestion. (d)~#4\end{answer}}
                {\begin{answer}\textbf{\thequestion. (#5)~Invalid answer choice.}\end{answer}}}}}
}

\setlength\parindent{0pt}
%usage \choice{ }{ }{ }{ }
%(A)(B)(C)(D)
\newcommand{\fourch}[5]{
    \par
    \begin{tabular}{*{4}{@{}p{0.23\textwidth}}}
        (a)~#1 & (b)~#2 & (c)~#3 & (d)~#4
    \end{tabular}
    \getanswer{#1}{#2}{#3}{#4}{#5}
}

%(A)(B)
%(C)(D)
\newcommand{\twoch}[5]{
    \par
    \begin{tabular}{*{2}{@{}p{0.46\textwidth}}}
        (a)~#1 & (b)~#2
    \end{tabular}
    \par
    \begin{tabular}{*{2}{@{}p{0.46\textwidth}}}
        (c)~#3 & (d)~#4
    \end{tabular}
    \getanswer{#1}{#2}{#3}{#4}{#5}
}

%(A)
%(B)
%(C)
%(D)
\newcommand{\onech}[5]{
    \par
    (a)~#1 \par (b)~#2 \par (c)~#3 \par (d)~#4
    \getanswer{#1}{#2}{#3}{#4}{#5}
}

\newlength\widthcha
\newlength\widthchb
\newlength\widthchc
\newlength\widthchd
\newlength\widthch
\newlength\tabmaxwidth

\setlength\tabmaxwidth{0.96\textwidth}
\newlength\fourthtabwidth
\setlength\fourthtabwidth{0.25\textwidth}
\newlength\halftabwidth
\setlength\halftabwidth{0.5\textwidth}

\newcommand{\choice}[5]{%
\settowidth\widthcha{AM.#1}\setlength{\widthch}{\widthcha}%
\settowidth\widthchb{BM.#2}%
\ifdim\widthch<\widthchb\relax\setlength{\widthch}{\widthchb}\fi%
    \settowidth\widthchb{CM.#3}%
\ifdim\widthch<\widthchb\relax\setlength{\widthch}{\widthchb}\fi%
    \settowidth\widthchb{DM.#4}%
\ifdim\widthch<\widthchb\relax\setlength{\widthch}{\widthchb}\fi%

% These if statements were bypassing the \onech option.
% \ifdim\widthch<\fourthtabwidth
%     \fourch{#1}{#2}{#3}{#4}{#5}
% \else\ifdim\widthch<\halftabwidth
% \ifdim\widthch>\fourthtabwidth
%     \twoch{#1}{#2}{#3}{#4}{#5}
% \else
%      \onech{#1}{#2}{#3}{#4}{#5}
%  \fi\fi\fi}

% Allows for the \onech option.
\ifdim\widthch>\halftabwidth
    \onech{#1}{#2}{#3}{#4}{#5}
\else\ifdim\widthch<\halftabwidth
\ifdim\widthch>\fourthtabwidth
    \twoch{#1}{#2}{#3}{#4}{#5}
\else
    \fourch{#1}{#2}{#3}{#4}{#5}
\fi\fi\fi}



\begin{document}

\begin{center}
  \bfseries\large
  Sylhet Cadet College

\normalsize
  Progress Test Examination - 2022

  Class: HSC

  Subject: Statistics 2nd Paper (MCQ)

  Time: 20 minutes \qquad \qquad \qquad \qquad Sub Code: 130  \qquad \qquad \qquad  \qquad Full Marks: 15

%  \normalfont\normalsize
 % 11.45a.m.~--~1.45p.m.
\end{center}

\textbf{Answer all the questions. Each question is worth one (1) mark.}

\begin{questions}

\question Mutually exclusive events are
\choice{always independent}{always dependent}{the relationship cannot be determined}{dependent or independent, depending on scenario}{b}

\question $P(A \cup B) = P(A) + P(B)$ is true for
\choice{independent events}{dependent events}{mutually exclusive events}{complementary events}{c}

\question If a coin is tossed $n$ times, how many outcomes are generated?
\choice{$n$}{$n^2$}{$2^n$}{$2n$}{c}


\question If $P(A) = \frac 18, P(A|B)=\frac14$, and $P(B|A)=\frac16, P(A\cap B)=?$

\choice{$\frac{1}{48}$}{$\frac{1}{24}$}{$\frac{1}{32}$}{1}{a}

\question A card is drawn at random from a well-shuffled deck of 52 cards. What is the probability that the drawn card is not a Queen?

\choice{$\frac 1 {52}$}{$\frac {4}{52}$}{$\frac{1}{13}$}{$\frac{12}{13}$}{a}

\question If X denotes number of successes in a coin toss, how many possible possible values of X are there?
\choice{0}{1}{2}{3}{a}

\question Which one is a correct condition of a pdf?
\choice {$\displaystyle \int_0^1 f(x) dx =1$}{$P(X)\ge 0$}{$\displaystyle \int_a^b f(x) dx =1 ; a\le x \le b$}{$\displaystyle \int_0^{Median} f(x) dx =0.55$}{a}

\question $P(A\cap \bar B) = ?$
\choice{$P(A)- P(A \cap B)$}{$P(B)- P(A \cap B)$}{$P(A)- P(A \cup B)$}{$P(B)- P(A \cup B)$}{a}

\textbf{Answer the questions 9-10 according to the following information.}

$P(x,y)=\frac{x+2y}{16}$; x = 0, 1 \& y = 0, 1, 2, 3

\question $P(X)=?$
\choice{$\frac{x+2y}{3}$}{$\frac{2x+y}{3}$}{$\frac{2x+3y}{3}$}{$\frac{x+3}{4}$}{d}

\question $P(X|Y=0)=?$
\choice{$\frac{x+2y}{4y+1}$}{$1$}{$x$}{0}{c}

\question If the mean of a Poisson distribution is 4, what is its variance?
\choice{2}{3}{4}{16}{a}

\question What is true of Poisson distribution?
\choice{$Mean > Variance$}{$Mean < Variance$}{$Mean = Variance^2$}{$Mean = Variance$}{a}

\question The Poisson distribution - 

\begin{enumerate}[i]
  \item is a discrete distribution
  \item gives a probability mass function
  \item gives a probability density function
\end{enumerate}

Which one is true?

\choice {i \& ii}{i \& iii}{i, ii, \& iii}{ii \& iii}{a}

\question Which formula represents the exponential growth?
\choice {$P_n =P_oe^{rn}$}{$P_n=P_o(1+r)^n$}{$P_n =P_one^{r}$}{$P_o =P_ne^{rn}$}{a}

\question Crude death rate is - 
\choice{$\frac{B}{F_{15-49}}\times 1000$}{$\frac{B}{P}\times 1000$}{$\frac D P \times 1000$}{$\frac B A \times 1000$}{a}

%\question To complete the song, the last answer should be
%\choice{a}{b}{c}{d}{e} % Invalid answer choice

\end{questions}

%\newpage  %Uncomment to put on new age
\bigskip

%\begin{multicols}{3}
%[
%Answer Key: (Correction required for 4 thru last)
%]
%\showallanswers % Phil Hirschorn
%\end{multicols}


\end{document}