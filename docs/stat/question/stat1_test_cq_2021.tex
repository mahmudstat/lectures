\documentclass{article}
\usepackage{geometry}
\usepackage{amsfonts}

\geometry{
a4paper, total={170mm, 257mm},left=20mm,
top=20mm,
}

\begin{document}

\begin{center}
  \bfseries\large
  Pabna Cadet College

\normalsize
  Test Examination - 2021

  Class: XII

  Subject: Statistics First Paper (Creative)

  Time: 2 hours \& 35 minutes \qquad \qquad Subject Code: 129  \qquad  \qquad Full Marks: 50

%  \normalfont\normalsize
 % 11.45a.m.~--~1.45p.m.
\end{center}

\noindent
\begin{tabular}{p{\dimexpr\linewidth-2\tabcolsep}}
  Answer five questions taking at least 2 (two) from each group. Figures in the right indicate full marks.\\
  \hline
\end{tabular}

\begin{center}
\textbf{Group A}
\end{center}

\begin{enumerate}
  \item
  \textbf{Prices and respective profits of some products are collected:}
 

\begin{table}[h]
 \begin{center}
\begin{tabular}{l|l|l|l}

Price (x)  & 20 & 30 & 25 \\ \hline
Profit (y) & 4  & 2  & 5  \\ 
\end{tabular}
\end{center}
\end{table}


  \begin{enumerate}
    \item
	What is a discrete variable? \hfill 1
    \item
    	Can fractional numbers be discrete? Explain briefly.  \hfill 2
    \item
    	Are, in the stem, $\displaystyle \sum_{i=1}^{n} \sum_{i=1}^{n} x_iy_j = \sum_{i=1}^{n} x_iy_i?$ Show statistically. \hfill 3
     \item
     	Prove empirically that sum of square is greater than square of sum of numbers. \hfill 4
  \end{enumerate}
  
    \item
  \textbf{In the test examination, marks of 10 students in statistics are \\ 69, 85, 77, 66, 92, 42, 97, 53, 83, 81.}
  \begin{enumerate}
    \item
	What is central tendency? \hfill 1
    \item
	When is median better than arithmetic mean? Explain with an example. \hfill 2
    \item  
	Find the 3rd the quartile, 5th septile, and 61st percentile from the data and explain.  \hfill 3
    \item
	Below which mark lie marks of 60\% students? Find using two separate quantiles. \hfill 4
\end{enumerate}

  \item
  \textbf{Marks of students of two colleges are:}
\begin{table}[h]
 \begin{center}
\begin{tabular}{l|l|l}
    & Studnets & AM \\ \hline
PCC & 8        & 82 \\ \hline
RCC & 10       & 80
\end{tabular}
\end{center}
\end{table}

where AM = Arithmetic Mean

  \begin{enumerate}
    \item
	What is change of origin? \hfill 1
    \item
	Does median depend on change of origin? Show statistically with a made-up data \hfill 2
    \item  
	Determine the combined mean of the students. \hfill 3
    \item
	If marks of 12 students of JCC are included to get a combined AM of 78.53, \\ what is the AM of marks of JCC students? \hfill 4
\end{enumerate}

  \item
	  \textbf{We have the data:}
	
	$\quad f_1=2, f_2 = 4, f_3 = 6$

	$\quad x_1 = -3, x_2 =7, x_3 = 4$

  \begin{enumerate}
    \item
	What is nominal data? \hfill 1
    \item
	Prove $\displaystyle \sum_{i=1}^{n} (x_i+c)=\sum_{i=1}^{n} x_i + nc$ \hfill 2
    \item  
	Show empirically that $\displaystyle \sum_{i=1}^{3} \sum_{j=1}^{3}(f_i-x_j) \ne \sum_{i=1}^{3} (f_i-x_i)$ \hfill 3
    \item
	Does summation depend on change of origin and scale? Show with an example\\ (using the given data).  \hfill 4
\end{enumerate}

\begin{center}
\textbf{Group B}
\end{center}

 \item
	  \textbf{The first four moments around 3 of a data are found to be -1, 5, 20, and 90.  }
  \begin{enumerate}
    \item
	What is skewness? \hfill 1
    \item
	Draw shapes of different types of skewness. \hfill 2
    \item  
	Determine the 2nd and 3rd central moments from the stem. \hfill 3
    \item
	Find the formulae of 4th and 5th central moments in terms of raw moments. \hfill 4
\end{enumerate}

 \item
	  \textbf{Diameter, height and volume for Black Cherry trees are obtained (from R Database)}
	
\begin{table}[h]
 \begin{center}
\begin{tabular}{l|l|l}
\hline
Girth & Height & Volume \\
8.3   & 70     & 10.3   \\ 
8.6   & 65     & 10.3   \\ 
8.8   & 63     & 10.2   \\ 
10.5  & 72     & 16.4   \\ 
10.7  & 81     & 18.8   \\ 
10.8  & 83     & 19.7   \\  \hline
\end{tabular}
\end{center}
\end{table}
  
  \begin{enumerate}
    \item
	What are moments? \hfill 1
    \item
	Which moment is equal to the variance? Show mathematically. \hfill 2
    \item  
	Find the first and second raw moments about 2 of volume of the trees. \hfill 3
    \item
	What do you understand of the possible shape of the data on height and volume of the trees? \hfill 4
\end{enumerate}

 \item
	  \textbf{Income of a freelancer in 6 successive months (from Jan to Jun) was found to be \\ 46.0, 49.5, 51.5, 50.6, 56.5, and 60 (in thousands BDT.).}
  \begin{enumerate}
    \item
	What is time series data? \hfill 1
    \item
	What are the components of a time series model? \hfill 2
    \item  
	Determine the 3-yearly moving average from the data. \hfill 3
    \item
	Draw the moving averages on a graph paper and interpret. \hfill 4
\end{enumerate}

  \item
	  \textbf{For proper management and functioning of an organization, gathering and analyzing \\ data are very crucial. Data play a key role in decision making. \\ Numerous Bangladeshi institutions gather data on production, revenue, education, health, and so on.}
  \begin{enumerate}
    \item
	What is official statistics? \hfill 1
    \item
	What do BBS do? \hfill 2
    \item  
	What are the sources of non-official statistics in Bangladesh? Explain in brief. \hfill 3
    \item
	Are adequate data collection, processing and analysis performed in Bangladesh? \\ Justify your opinion.  \hfill 4
\end{enumerate}
\end{enumerate}

\end{document}