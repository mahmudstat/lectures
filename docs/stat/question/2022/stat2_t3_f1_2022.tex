\documentclass{article}
\usepackage{geometry}
\usepackage{amsfonts}

\geometry{
a4paper, total={170mm, 257mm},left=20mm,
top=20mm,
}

\begin{document}

\begin{center}
  \bfseries\large
  Sylhet Cadet College

\normalsize
Fortnightly Examination - 2022

  Class: XII

  Subject: Statistics 2nd Paper

  Time: 40 minutes \qquad \qquad  \qquad \qquad Subject Code: 129  \qquad  \qquad \qquad  \qquad Full Marks: 20

%  \normalfont\normalsize
 % 11.45a.m.~--~1.45p.m.
\end{center}

\noindent
\begin{tabular}{p{\dimexpr\linewidth-2\tabcolsep}}
  Answer all the questions. Figures in the right indicate full marks.\\
  \hline
\end{tabular}

\begin{enumerate}
  \item
  \textbf{Define the following terms with example.}  \hfill 5

Experiment, sample space, event, simple event, mutually excluysive event. 

 \item
	  \textbf{A box contains four blue and 6 green balls. 3 balls are drawn randomly.} 
  
  \begin{enumerate}
    \item
	What is the value of $^nC_r$? \hfill 1
    \item
	Illustrate the difference between permutation and combination with an example. \hfill 2
    \item  
	What is the probability that all balls are green? \hfill 3
    \item
	What is the probabilith that one ball has a different color? \hfill 4
  \end{enumerate}

 \item
	  \textbf{Write down the conditions of the axiomatic approach of probability.} \hfill 3

 \item
	  \textbf{What is the probability that Grameenphone sim numbers starting with 017 end with 0?} \hfill 2	  \linebreak \linebreak \linebreak  \linebreak
	  
\end{enumerate}

\end{document}