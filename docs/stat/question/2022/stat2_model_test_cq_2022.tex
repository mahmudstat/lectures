\documentclass{article}
\usepackage{geometry}
\usepackage{amsfonts}
\usepackage{amsmath}

\geometry{
a4paper, total={170mm, 257mm},left=20mm,
top=20mm,
}

\begin{document}

\begin{center}
  \bfseries\large
  Sylhet Cadet College

\normalsize
  Model Test Examination - 2022

  Class: HSC

  Subject: Statistics 2nd Paper (Creative)

  Time: 1 hours \& 40 minutes \qquad \qquad Subject Code: 130  \qquad  \qquad Full Marks: 30

%  \normalfont\normalsize
 % 11.45a.m.~--~1.45p.m.
\end{center}

\noindent
\begin{tabular}{p{\dimexpr\linewidth-2\tabcolsep}}
  Answer three questions taking at least 1 (one) from each group. Figures in the right indicate full marks.\\
  \hline
\end{tabular}

\begin{center}
\textbf{Group A}
\end{center}

\begin{enumerate}

  \item
  \textbf{$P(A\vert B) = \frac 1 8, P(A) = \frac 12, P(B) = \frac 15$}
 
  \begin{enumerate}
    \item
	Write down the range of probability. \hfill 1
    \item
    	Find $P(A\cap B)$.  \hfill 2
    \item
    	Find $P(A\vert \bar B)$. \hfill 3
     \item
     	Are the probabilities $P(A\vert B)$ and $P(B\vert A)$ equal? Justify\hfill 4
  \end{enumerate}
  
  \item
  \textbf{It is observed that in a college, there are 100 students, of whom 30 play football, \\ 40 play cricket, and 20 play both.}
 
  \begin{enumerate}
    \item
	What is a sample space? \hfill 1
    \item
    	What is the relationship between independence and mutual excluvity?  \hfill 2
    \item
    	Are the probabilities of playing cricket and that of football independent? Prove. \hfill 3
     \item
     	If a student is selected randomly, and if he plays cricket, what is the probability that \\ he does not play football? \hfill 4
  \end{enumerate}
  
    \item
  \textbf{The joint probability function of two random variables X and Y is given below:}
  
  $\displaystyle P(X,Y) = \frac {x+2y}{16}; x = 0, 1; y = 0 ,1,2,3$
 
  \begin{enumerate}
    \item
	Write down the formula of conditional proibability. \hfill 1
    \item
    	What is the relationship between marginal and joint probability? \hfill 2
    \item
    	Find P(X). \hfill 3
     \item
     	Find $P(X\vert Y)$ and $P(X\vert 0)$. \hfill 4
  \end{enumerate}

 \item
  \textbf{The probability density function of a continuous random variable is}

$$
  f(x) =
\begin{cases}
k(x+1),  & 0 \le x \le 1 \\
0, & otherwise
\end{cases}
$$

  \begin{enumerate}
    \item
	What is a random variable? \hfill 1
    \item
    	Find the value of k \hfill 2
    \item
    	Find the probability that the values of x would lie between 0 and 0.5. \hfill 3
     \item
     	What is the probability that X is greater than 0.8?  \hfill 4
  \end{enumerate}

\begin{center}
\textbf{Group B}
\end{center}
  
      \item
  \textbf{Various sales and their probabilities of a grocery store is given below}
  
  \begin{table}[h]
  \centering
\begin{tabular}{llllll}
Sales & 200 & 250 & 275 & 310 & 350 \\
Probability & 0.10 & 0.20 & 0.40 & 0.25 & 0.05
\end{tabular}
\end{table}

  \begin{enumerate}
    \item
	Can the expectation of a random variable be negative? \hfill 1
    \item
    	Find the expected sales of the store on a given day. \hfill 2
    \item
    	Compute the dispersion of sales f the store. \hfill 3
     \item
     	To make the expected sale 280, what sale does the store need in place of 200? \hfill 4
  \end{enumerate}
  
      \item
  \textbf{$P(X) = \frac{3-\vert 4-x\vert}{k}; x=2,3,4,5,6$}

  \begin{enumerate}
    \item
	What is the Expectation equivalent to? \hfill 1
    \item
    	Find the value of k. \hfill 2
    \item
    	Determine the value of the expectation. \hfill 3
     \item
     	Find $V(2X-1)$ \hfill 4
  \end{enumerate}

  \item
  \textbf{In winter, the probability that it rains on a particular day is 0.015. An analyst observes \\ 100 winter days.}
 
  \begin{enumerate}
    \item
	What is an experiment? \hfill 1
    \item
    	When can the Poisson distribution be approximated by the Binomial distribution? \hfill 2
    \item
    	Find, using Binomial distribution, the probability that  it would not rain at all on the \\ observed days. \hfill 3
     \item
     	Find the probability in 3(c) using Poisson distribution.  \hfill 4
  \end{enumerate}

 \item
  \textbf{For projection of population in a future time period, demographers use simple, \\ geometric or exponential growth technique. Each method has its advantages and \\ disadvantages.}

  \begin{enumerate}
    \item
	What is geometric growth? \hfill 1
    \item
    	In geometric growth method, obtain the formula for time required for the population to get \\ doubled [denote rate as r]. \hfill 2
    \item
    	In exponential method, how much unit of time is required for the population to get tripled?  \hfill 3
     \item
     	For projecting (predicting future values), is geometric growth method better than the \\ exponential method? Justify.  \hfill 4
  \end{enumerate}
  
\end{enumerate}

\end{document}