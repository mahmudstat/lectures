\documentclass{article}
\usepackage{geometry}
\usepackage{amsfonts}

\geometry{
a4paper, total={170mm, 257mm},left=20mm,
top=20mm,
}

\begin{document}

\begin{center}
  \bfseries\large
  Sylhet Cadet College

\normalsize
  Model Test Examination - 2022

  Class: XII

  Subject: Statistics First Paper (Creative)

  Time: 1 hour \& 40 minutes \qquad \qquad \qquad Subject Code: 129  \qquad  \qquad \qquad Full Marks: 30

%  \normalfont\normalsize
 % 11.45a.m.~--~1.45p.m.
\end{center}

\noindent
\begin{tabular}{p{\dimexpr\linewidth-2\tabcolsep}}
  Answer three questions taking at least 1 (one) from each group. Figures in the right indicate full marks.\\
  \hline
\end{tabular}

\begin{center}
\textbf{Group A}
\end{center}

\begin{enumerate}

   \item
	  \textbf{Mean monthly salaries of employees of two companies A \& B are tk. 65,000 and tk. 75,000. The combined arithmetic mean (AM) is tk. 71,000 and number of employees in the company A is 20.} 
	  
  \begin{enumerate}
    \item
	Write down the formula of combined AM for n groups. \hfill 1
    \item
	What is the combined AM of two data sets with AM 35 and 45 and number values equal? \hfill 2
    \item  
	How many employees are there in the company B? \hfill 3
    \item
	Question \hfill 4
  \end{enumerate}

     \item
	  \textbf{A departmental store records their sales. An analysis of products with prices less than tk. 30 generates the following table.} 
	  
\begin{table}[h]
\centering
\begin{tabular}{ccccccc}
Price     & 0-5 & 5-10 & 10-15 & 15-20 & 20-25 & 25-30 \\ \hline
Frequency & 1   & 0    & 2     & 3     & 8     & 12   
\end{tabular}
\end{table}
  
  \begin{enumerate}
    \item
	What is relative frequency? \hfill 1
    \item
	If $Y = a + bX$, $\bar Y = ?$ \hfill 2
    \item  
	Find 67th Percentile and 3rd Quartile and explain. \hfill 3
    \item
	Is AM or Median more suitable for this data? Elucidate. \hfill 4
  \end{enumerate}
  
   \item
	  \textbf{Arithmetic and Harmonic Mean (HM) of two numbers are 25 and 9, respectively.} 
    \begin{enumerate}
    \item
	When is HM useful? \hfill 1
    \item
	Derive HM formula using the concept of average velocity. \hfill 2
    \item  
	Find the two values from the stem. \hfill 3
    \item
	Question \hfill 4
  \end{enumerate}
  
   \item
	  \textbf{Statement} 
    \begin{enumerate}
    \item
	Question \hfill 1
    \item
	Question \hfill 2
    \item  
	Question \hfill 3
    \item
	Question \hfill 4
  \end{enumerate}
 
\begin{center}
\textbf{Group B}
\end{center}

   \item
	  \textbf{Marks obtained by a student in 7 subjects are} 
	  \begin{center}
	  70, 66, 55, 45, 80, 30, 82
	\end{center}
  
  \begin{enumerate}
    \item
	What is negative skewness? \hfill 1
    \item
	Draw graphs of positive and negative skewness showing the locations of mean and median. \hfill 2
    \item  
	Determine the five number summary from the stem and explain. \hfill 3
    \item
	Are the data symmetric? If not, comment on the pattern of data. \hfill 4
\end{enumerate}
  
     \item
	  \textbf{Goals scored by Karim Benzema in five seasons are recorded to be the following:} 
	  
	  \begin{table}[h]
	  \centering
\begin{tabular}{c|c|c}
Season & La Liga (x) & Uefa Champions League (y) \\ \hline
2017-18 & 5 & 5 \\ 
2018-19 & 21 & 4 \\
2019-20 & 21 & 5 \\
2020-21 & 23 & 6 \\ 
2021-22 & 27 & 15 \\ \hline
\end{tabular}
\end{table}
  
  \begin{enumerate}
    \item
	What is a quantitative variable? \hfill 1
    \item
	What is the notation to denote his total number of goals? \hfill 2
    \item  
	Compute $\displaystyle \sum_{i=1}^5 (y_i - 3)^2$ \hfill 
    \item
	Find total number of goals using two different notations and examine whether they match. \hfill 4
  \end{enumerate}
  
     \item
	  \textbf{Given below is a series of data.} 
	  
	  	    \begin{center}
	 $5, 7, 9, \cdots , 123$
	    \end{center}
  
  \begin{enumerate}
    \item
	What is the summation of natural numbers up to nth value? \hfill 1
    \item
	Find the arithmetic mean of natural numbers from 1 up to 20. \hfill 2
    \item  
	Find the arithmetic mean of the given series. \hfill 3
    \item
	Prove that arithmetic mean is greater than gemetric mean theoretically and empricially. \hfill 4
  \end{enumerate}
  
   \item
	  \textbf{In ODI cricket, two top batsmen are (as of 2nd Sept, 2022) Babar Azam and Rassie van der Dussen. Their average (arithmetic mean) scores are 59.79 and 69.32, appearing in 90 (including being not out in 12 occassions) and 33 (including being not out in 11 occassions) matches, respectively.} 
  
  \begin{enumerate}
    \item
	When is arithmetic mean inappropriate to use? \hfill 1
    \item
	Is arithmetic mean always suitable for comparison? \hfill 2
    \item  
	Find the combined arithmetic mean and explain. \hfill 3
    \item
	How to compare two sets of data having significantly distinct ranges? \hfill 4
  \end{enumerate}

\end{enumerate}

\end{document}