\documentclass{article}
\usepackage{geometry}
\usepackage{amsfonts}

\geometry{
a4paper, total={170mm, 257mm},left=20mm,
top=20mm,
}

\begin{document}

\begin{center}
  \bfseries\large
  Sylhet Cadet College

\normalsize
  Pre-Test Examination - 2022

  Class: XII
  
  Set - B

  Subject: Statistics First Paper (Creative)

  Time: 2 hour \& 35 minutes \qquad \qquad \qquad Subject Code: 129  \qquad  \qquad \qquad Full Marks: 30

%  \normalfont\normalsize
 % 11.45a.m.~--~1.45p.m.
\end{center}

\noindent
\begin{tabular}{p{\dimexpr\linewidth-2\tabcolsep}}
  Answer FIVE questions taking at least 1 (one) from each group. Figures in the right indicate full marks.\\
  \hline
\end{tabular}

\begin{center}
\textbf{Group A}
\end{center}

\begin{enumerate}

   \item
	  \textbf{Mean monthly salaries of employees of two companies A \& B are tk. 65,000 and tk. 75,000. The combined arithmetic mean (AM) is tk. 71,000 and number of employees in the company A is 20.} 
	  
  \begin{enumerate}
    \item
	Write down the formula of combined AM for k groups. \hfill 1
    \item
	What is the combined AM of two data sets with AM 35 and 45 and number of values equal? \hfill 2
    \item  
	How many employees are there in the company B? \hfill 3
    \item
	Salary of an employee of company A was recorded as tk. 60,000 in place of 65,000. What is the new AM of company A. Also find the corrected combined AM.\hfill 4
  \end{enumerate}

     \item
	  \textbf{A departmental store records their sales. An analysis of products with prices less than tk. 30 generates the following table.} 
	  
\begin{table}[h]
\centering
\begin{tabular}{ccccccc}
Price     & 0-5 & 5-10 & 10-15 & 15-20 & 20-25 & 25-30 \\ \hline
Frequency & 1   & 0    & 2     & 3     & 8     & 12   
\end{tabular}
\end{table}
  
  \begin{enumerate}
    \item
	What is relative frequency? \hfill 1
    \item
	If $Y = a + bX$, $\bar Y = ?$ \hfill 2
    \item  
	Find 67th Percentile and 3rd Quartile and explain. \hfill 3
    \item
	Is AM or Median more suitable for this data? Elucidate. \hfill 4
  \end{enumerate}
  
   \item
	  \textbf{Arithmetic and Harmonic Mean (HM) of two numbers are 25 and 9, respectively.} 
    \begin{enumerate}
    \item
	When is HM useful? \hfill 1
    \item
	Derive HM formula using the concept of average velocity. \hfill 2
    \item  
	Find the two values from the stem. \hfill 3
    \item
	Show mathematically that $HM \le AM$ (for n=2) \hfill 4
  \end{enumerate}
  
   \item
	  \textbf{There has been an increase in average lifetime of people of Bangladesh. To get more insight on this, a research was conducted, in which ages of retired government employees were recorded. A sample of 10 people is given below:}
	  
	  \begin{center}
	  75, 62, 63, 72, 66, 76, 59, 77, 70, 79
	  \end{center}
    \begin{enumerate}
    \item
	What is the 2nd central moment? \hfill 1
    \item
	Why is the first central moment zero? \hfill 2
    \item  
	Find the variance of the data. \hfill 3
    \item
	Are the data symmetric? Justify. \hfill 4
  \end{enumerate}
 
\begin{center}
\textbf{Group B}
\end{center}

   \item
	  \textbf{Marks obtained by a student in 7 subjects are} 
	  \begin{center}
	  70, 66, 55, 45, 80, 30, 82
	\end{center}
  
  \begin{enumerate}
    \item
	What is negative skewness? \hfill 1
    \item
	Draw graphs of positive and negative skewness showing the locations of mean and median. \hfill 2
    \item  
	Determine the five number summary from the stem and explain. \hfill 3
    \item
	Analyze symmetry of the data by drawing the box plot. \hfill 4
\end{enumerate}
  
  
     \item
	  \textbf{Given below is a series of data.} 
	  
	  	    \begin{center}
	 $5, 7, 9, \cdots , 123$
	    \end{center}
  
  \begin{enumerate}
    \item
	What is the summation of natural numbers up to nth value? \hfill 1
    \item
	Find the arithmetic mean of natural numbers from 1 up to 20. \hfill 2
    \item  
	Find the arithmetic mean of the given series. \hfill 3
    \item
	Prove that arithmetic mean is greater than gemetric mean theoretically and empricially. \hfill 4
  \end{enumerate}
  
   \item
	  \textbf{The yearly revenue (in hundred thousand) of shoe manufacturer company is given below} 
  \begin{table}[h]
\centering
\begin{tabular}{cccccccc}
Year     & 2005 & 2006 & 2007 & 2008 & 2009 & 2010 & 2011 \\ \hline
Revenue & 35   & 22    & 40     & 35     & 50     & 42 & 60   
\end{tabular}
\end{table}
  
  \begin{enumerate}
    \item
	What is general trend? \hfill 1
    \item
	Which method of determining trend gives only two values? \hfill 2
    \item  
	Determine the trend using three-yearly moving average method. \hfill 3
    \item
	Find the trend using graphical method and extrapolate the approximate revenue earned in 2012. \hfill 4
  \end{enumerate}

 \item
	  \textbf{A farmer in Dinajpur district produces seasonal crops. First four moments around 9 of his daily earnings are computed as -1, 8, -16, and 25.}
  
  \begin{enumerate}
    \item
	What is the Box and Whisker plot? \hfill 1
    \item
	Can Box and Whisker plot suggest symmetry? \hfill 2
    \item  
	 Find the arithmetic mean and variance of the farmer's earnings.\hfill 3
    \item
	Do the earnings produce a symmetric data? Analyze. \hfill 4
  \end{enumerate}

\end{enumerate}

\end{document}