\documentclass{exam}
%\documentclass[11pt,a4paper]{exam}
\usepackage{amsmath,amsthm,amsfonts,amssymb,dsfont}
\usepackage{ifthen}
\usepackage{enumerate}% http://ctan.org/pkg/enumerate
\usepackage{multicol}



% Accumulate the answers. Unmodified from Phil Hirschorn's answer
% https://tex.stackexchange.com/questions/15350/showing-solutions-of-the-questions-separately/15353
\newbox\allanswers
\setbox\allanswers=\vbox{}

\newenvironment{answer}
{%
    \global\setbox\allanswers=\vbox\bgroup
    \unvbox\allanswers
}%
{%
    \bigbreak
    \egroup
}

\newcommand{\showallanswers}{\par\unvbox\allanswers}
% End Phil's answer


% Is there a better way?
\newcommand*{\getanswer}[5]{%
    \ifthenelse{\equal{#5}{a}}
    {\begin{answer}\thequestion. (a)~#1\end{answer}}
    {\ifthenelse{\equal{#5}{b}}
        {\begin{answer}\thequestion. (b)~#2\end{answer}}
        {\ifthenelse{\equal{#5}{c}}
            {\begin{answer}\thequestion. (c)~#3\end{answer}}
            {\ifthenelse{\equal{#5}{d}}
                {\begin{answer}\thequestion. (d)~#4\end{answer}}
                {\begin{answer}\textbf{\thequestion. (#5)~Invalid answer choice.}\end{answer}}}}}
}

\setlength\parindent{0pt}
%usage \choice{ }{ }{ }{ }
%(A)(B)(C)(D)
\newcommand{\fourch}[5]{
    \par
    \begin{tabular}{*{4}{@{}p{0.23\textwidth}}}
        (a)~#1 & (b)~#2 & (c)~#3 & (d)~#4
    \end{tabular}
    \getanswer{#1}{#2}{#3}{#4}{#5}
}

%(A)(B)
%(C)(D)
\newcommand{\twoch}[5]{
    \par
    \begin{tabular}{*{2}{@{}p{0.46\textwidth}}}
        (a)~#1 & (b)~#2
    \end{tabular}
    \par
    \begin{tabular}{*{2}{@{}p{0.46\textwidth}}}
        (c)~#3 & (d)~#4
    \end{tabular}
    \getanswer{#1}{#2}{#3}{#4}{#5}
}

%(A)
%(B)
%(C)
%(D)
\newcommand{\onech}[5]{
    \par
    (a)~#1 \par (b)~#2 \par (c)~#3 \par (d)~#4
    \getanswer{#1}{#2}{#3}{#4}{#5}
}

\newlength\widthcha
\newlength\widthchb
\newlength\widthchc
\newlength\widthchd
\newlength\widthch
\newlength\tabmaxwidth

\setlength\tabmaxwidth{0.96\textwidth}
\newlength\fourthtabwidth
\setlength\fourthtabwidth{0.25\textwidth}
\newlength\halftabwidth
\setlength\halftabwidth{0.5\textwidth}

\newcommand{\choice}[5]{%
\settowidth\widthcha{AM.#1}\setlength{\widthch}{\widthcha}%
\settowidth\widthchb{BM.#2}%
\ifdim\widthch<\widthchb\relax\setlength{\widthch}{\widthchb}\fi%
    \settowidth\widthchb{CM.#3}%
\ifdim\widthch<\widthchb\relax\setlength{\widthch}{\widthchb}\fi%
    \settowidth\widthchb{DM.#4}%
\ifdim\widthch<\widthchb\relax\setlength{\widthch}{\widthchb}\fi%

% These if statements were bypassing the \onech option.
% \ifdim\widthch<\fourthtabwidth
%     \fourch{#1}{#2}{#3}{#4}{#5}
% \else\ifdim\widthch<\halftabwidth
% \ifdim\widthch>\fourthtabwidth
%     \twoch{#1}{#2}{#3}{#4}{#5}
% \else
%      \onech{#1}{#2}{#3}{#4}{#5}
%  \fi\fi\fi}

% Allows for the \onech option.
\ifdim\widthch>\halftabwidth
    \onech{#1}{#2}{#3}{#4}{#5}
\else\ifdim\widthch<\halftabwidth
\ifdim\widthch>\fourthtabwidth
    \twoch{#1}{#2}{#3}{#4}{#5}
\else
    \fourch{#1}{#2}{#3}{#4}{#5}
\fi\fi\fi}


\begin{document}

\begin{center}
  \bfseries\large
  Sylhet Cadet College

\normalsize
  Pre-Test Examination - 2022

  Class: XII
  
    Set - C

  Subject: Statistics First Paper (MCQ)

  Time: 25 minutes \qquad \qquad \qquad \qquad Subject Code: 129   \qquad \qquad \qquad  \qquad Full Marks: 25

%  \normalfont\normalsize
 % 11.45a.m.~--~1.45p.m.
\end{center}

\textbf{Answer all the questions. Each question is worth one (1) mark.}  

\begin{questions}

\question \textbf{Which measure of central tendencyis suitable for qualitative variable?}
\choice{Arithmetic Mean}{Harmonic Mean}{Quadratic Mean}{Mode}{d}

\question \textbf{In presence of negative values, which measure is not usable?}
\choice{Arithmetic Mean}{Geometric Mean}{Quadratic Mean}{Harmonic Mean}{b}

\question \textbf{What is the arithmetic mean of first n odd natural numbers?}
\choice{$\frac{n+1}{n}$}{n}{n+1}{$\frac{n+1}{2}$}{b}

\question \textbf{Which measure is not used in determining skewness?}
\choice{Arithmetic Mean}{Geometric Mean}{Median}{Mode}{b}

\question \textbf{Inappropriate for algebraic analysis--}

i. Median \\
ii. Mode \\
iii. Geometric Mean

Which one is true?

\choice{i}{ii}{i \& ii}{ii \& iii}{a}

\textbf{Answer the next two questions based on the following information}

  \begin{table}[h]
\centering
\begin{tabular}{cccccc}
Accident     & 4 & 6 & 7 & 8 & 9\\ \hline
Frequency & 2   & 0    & 4     & 4     & 1   
\end{tabular}
\end{table}
\question \textbf{Fifth Decile is --}
\choice{0}{8}{7}{6}{c}

\question \textbf{Which of the following is mode?}
\choice{4}{8}{0}{7}{b}

\question \textbf{Which measure gives a value from within the values?}
\choice{Arithmetic Mean}{Geometric Mean}{Median}{Mode}{d}

\question \textbf{Which one is not a proper measure of central tendency?}
\choice{2nd Quartile}{Third Decile}{3rd Quintile}{110th Percentile}{d}

\question \textbf{Which can be used to measure dispersion?}
\choice{$\mu_2'$}{$\mu_1$}{$\mu_2$}{$\mu_1'$}{c}

\question \textbf{The formula of coefficient of variance (CV) is --}
\choice{$\frac{\mu_2}{n}\times 100$}{$\frac{\mu_2}{\mu_1}\times 100$}{$\frac{\mu_2}{\bar x}\times 100$}{$\frac{\mu_3}{\sigma}\times 100$}{c}

\question \textbf{First moment around zero is --}
\choice{0}{1}{-1}{Arithmetic Mean}{a}

\question \textbf{Which values are used in constructing Box \& Whisker Plot?}
\choice{Mode}{$X_L$}{$Q_1 \& Q_3$}{$Q_1, Q_2 \& Q_3$}{a}

\question \textbf{Which might have a negative value?}
\choice{$\mu_4$}{$\mu_3$}{$\mu_2'$}{$\mu_2$}{b}

\question \textbf{In a symmatric distribution--}

i. Arithmetic Mean = Mode = Median \\
ii. $Q_2-Q_1 = Q_3-Q_2$  \\
iii. $Q_1-X_H = X_H-Q_3$

Which one is true?

\choice{i \& ii}{ii \& iii}{i \&iii}{i, ii \&iii}{d}

\question \textbf{For a data, $Q_3=41.6, Q_1=17.2, Median = 29, \& AM = 30$; What is Coefficient of skewness?}
\choice{24.4}{1}{0.03}{29.45}{d}

\question \textbf{$\sqrt{\beta_1}=-0.23$ implies--}
\choice{Left Skew}{Symmetry}{Right Skew}{Mesokurtic}{a}

\question \textbf{Which is not included in five number summary?}
\choice{Arithmetic Mean}{$X_H$}{$Q_2$}{$Q_3$}{a}

\question \textbf{$\beta_2 = \sqrt 9$ implies data are--}
\choice{Leptokurtic}{Platykurtic}{Mesokurtic}{Symmetric}{c}

\question \textbf{2nd Central Moment is --}
\choice{$\mu_2-\mu_1'$}{$\mu_2+\mu_1'$}{$\mu_2-\mu_1'^2$}{$\mu_2'-\mu_1^2$}{c}

\question \textbf{A company is constantly getting greater revenue than previous year; this is--}
\choice{Seasonal Variation}{General Trend}{Irregular Variation}{Cyclic Variation}{b}

\question \textbf{Which is not a method of finding general trend?}
\choice{Graphical Method}{Moving Average}{Semi-Average}{Moving Median}{d}

\textbf{Answer the next two questions based on the following table:}

  \begin{table}[h]
\centering
\begin{tabular}{ccccccc}
Year     & 2007 & 2008 & 2009 & 2010 & 2011 & 2012\\ \hline
Sales & 5   & 35    & 34     & 40     & 42  & 204 
\end{tabular}
\end{table}

\question \textbf{In Semi-Average method, what is the 2nd average?}
\choice{74}{24.67}{95.33}{28}{c}

\question \textbf{For this data, which method would give the best measure of trend?}
\choice{3-yearly Moving Average}{4-yearly Moving Average}{Semi-Average}{Graphical Method}{a}

\question \textbf{which component of time series represents a natural disaster?}
\choice{Seasonal Variation}{General Trend}{Irregular Variation}{Cyclic Variation}{c}

%\question \textbf{To complete the song, the last answer should be
%\choice{a}{b}{c}{d}{e} % Invalid answer choice

\end{questions}

%\newpage  %Uncomment to put on new age
\bigskip

%\begin{multicols}{3}
%[
%Answer Key: (Correction required)
%]
%\showallanswers % Phil Hirschorn
%\end{multicols}


\end{document}