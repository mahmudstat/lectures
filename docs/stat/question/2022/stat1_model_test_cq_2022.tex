\documentclass{article}
\usepackage{geometry}
\usepackage{amsfonts}

\geometry{
a4paper, total={170mm, 257mm},left=20mm,
top=20mm,
}

\begin{document}

\begin{center}
  \bfseries\large
  Sylhet Cadet College

\normalsize
  Model Test Examination - 2022

  Class: HSC

  Subject: Statistics First Paper (Creative)

  Time: 1 hour \& 40 minutes \qquad \qquad \qquad Subject Code: 129  \qquad  \qquad \qquad Full Marks: 30

%  \normalfont\normalsize
 % 11.45a.m.~--~1.45p.m.
\end{center}

\noindent
\begin{tabular}{p{\dimexpr\linewidth-2\tabcolsep}}
  Answer three questions taking at least 1 (one) from each group. Figures in the right indicate full marks.\\
  \hline
\end{tabular}

\begin{center}
\textbf{Group A}
\end{center}

\begin{enumerate}
  \item
  \textbf{Call duration of 6 calls in a customer care center are}
  
  \begin{center}
  2, 2.5, 1.5, 5, 6, 3
  \end{center}
 
  \begin{enumerate}
    \item
	What is a sample? \hfill 1
    \item
    	Are all quantitative variables continuous?  \hfill 2
    \item
    	Determine $\displaystyle \sum_{i=1}^7 (x_i-3)^3$ \hfill 3
     \item
     	Find the values of  $\displaystyle \sum_{i=1}^7 (x_i-5)^2$ and $\displaystyle \sum_{i=1}^7 x_i^2+5.$  \hfill 4 \\
     	Explain mathematically why they are unequal.
  \end{enumerate}
  

    \item
  \textbf{Scores of a batsman in the last 20 innings are} 
   \begin{center}
  	 28, 30, 16, 48, 50, 86, 105, 20, 10, 36, \\
  	 12, 25, 20, 35, 65, 12, 10, 76, 55, 32
  	 \end{center}
  \begin{enumerate}
    \item
	Write down the formula of weighted harmonic mean \hfill 1
    \item
	Can median be a better measure of central tendency than arithmetic mean for this data?  \hfill 2
    \item  
	Draw a stem and leaf plot from the data and explain.  \hfill 3
    \item
	Make a frequency distribution from the data and also find and interpret cumulative  \hfill 4 \\ frequencies and percentages.
\end{enumerate}

    \item
  \textbf{Frequency distribution of marks in statistics of a college is given in the following table.}
 

\begin{table}[h]
\centering
\begin{tabular}{ccc}
\hline
Marks & \begin{tabular}[c]{@{}c@{}}Number of Students\\ Group - A\end{tabular} & \begin{tabular}[c]{@{}c@{}}Number of Students\\ Group - B\end{tabular} \\ \hline
25-30 & 11 & 10 \\ 
30-35 & 18 & 16 \\ 
35-40 & 21 & 22 \\ 
40-45 & 26 & 28 \\ 
45-50 & 14 & 9 \\ \hline
\end{tabular}
\end{table}

  \begin{enumerate}
    \item
	What is data \hfill 1
    \item
	What are the disadvantages of secondary data? \hfill 2
    \item  
	Calculate the arithmetic mean of Group - A \hfill 3
    \item
	Compute the combined mean. Is it greather than the arithmetic mean of Group - B? Explain the possible reason(s). \hfill 4
\end{enumerate}

   \item
  \textbf{In the test examination, marks of 10 students in statistics are }  
  \begin{center}
69, 85, 77, 66, 92, 42, 97, 53, 83, 81. 
  \end{center}
  \begin{enumerate}
    \item
	What is central tendency? \hfill 1
    \item
	When is median better than arithmetic mean? Explain with an example. \hfill 2
    \item  
	Find the  and 63rd percentile from the data and explain.  \hfill 3
    \item
	Below which mark lie marks of 70\% students? Find using two separate quantiles. \hfill 4
\end{enumerate}

\begin{center}
\textbf{Group B}
\end{center}

 \item
	  \textbf{Marks obtained by a student in 7 subjects are} 
	  \begin{center}
	  70, 66, 55, 45, 80, 30, 82
	\end{center}
  
  \begin{enumerate}
    \item
	What is negative skewness? \hfill 1
    \item
	Draw graphs of positive and negative skewness showing the locations of mean and median. \hfill 2
    \item  
	Determine the five number summary from the stem and explain. \hfill 3
    \item
	Are the data symmetric? If not, comment on the pattern of data. \hfill 4
\end{enumerate}

 \item
	  \textbf{Income of a freelancer in 6 successive months (from Jan to Jun) was found to be \\ 46.0, 49.5, 51.5, 50.6, 56.5, and 60 (in thousands BDT.).}
  \begin{enumerate}
    \item
	What is time series data? \hfill 1
    \item
	What are the components of a time series model? \hfill 2
    \item  
	Determine the 3-monthly moving average from the data. \hfill 3
    \item
	Draw the moving averages on a graph paper and interpret. \hfill 4
\end{enumerate}

 \item
	  \textbf{In a distribution, $\bar X = 70, Me = 75, Q_1 = 40, Q_3 = 63, \sigma = 2,$ and $\mu_3 = 5.77$.}\
  \begin{enumerate}
    \item
	What is skewness? \hfill 1
    \item
	Can the second central moments be negative? \hfill 2
    \item  
	Find the skewness with Bowley's formula from the data and explain. \hfill 3
    \item
	Estimate the skewness using the method of moment. Does it comply with Bowley's estimate? \hfill 4
\end{enumerate}

 \item
	  \textbf{Annual sales of company are as given in the following}\
	  
	  \begin{table}[h]
	  \centering
\begin{tabular}{l|l|l|l|l|l|l|l}
Year & 2010 & 2011 & 2012 & 2013 & 2014 & 2015 & 2016 \\ \hline
Profit (million) & 40 & 45 & 46 & 53 & 65 & 70 & 73
\end{tabular}
\end{table}

  \begin{enumerate}
    \item
	What is a trend? \hfill 1
    \item
	Do the data in the stem seem to have a trend? \hfill 2
    \item  
	Find the trend using semi-average method. \hfill 3
    \item
	Find the trend using 2-yearly moving average method. Would it better if we used 3-yearly  \hfill 4 \\  method?
\end{enumerate}

\end{enumerate}

\end{document}