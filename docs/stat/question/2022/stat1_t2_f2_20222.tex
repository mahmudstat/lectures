\documentclass{article}
\usepackage{geometry}
\usepackage{amsfonts}

\geometry{
a4paper, total={170mm, 257mm},left=20mm,
top=20mm,
}

\begin{document}

\begin{center}
  \bfseries\large
  Sylhet Cadet College

\normalsize
Fortnightly Examination - 2022

  Class: XI

  Subject: Statistics First Paper

  Time: 40 minutes \qquad \qquad  \qquad \qquad Subject Code: 129  \qquad  \qquad \qquad  \qquad Full Marks: 15

%  \normalfont\normalsize
 % 11.45a.m.~--~1.45p.m.
\end{center}

\noindent
\begin{tabular}{p{\dimexpr\linewidth-2\tabcolsep}}
  Answer all the questions. Figures in the right indicate full marks.\\
  \hline
\end{tabular}

\begin{enumerate}


	\item
	\textbf{The arithmetic mean of first six values of a data set is 40 and that of first four numbers is 20. What is the arithmetic mean of the last two numbers?} \hfill 3 \\

	\item
	\textbf{Find the 3rd quartile and 64th percentile from the followingh data and interpret.} \hfill 2
  \begin{center}	
15, 13,  9, 22,  2, 30, 11,  8, 27, 25, 16
  \end{center}
  
      \item
  \textbf{Frequency distribution of marks in statistics of a college is given in the following table.}
 

\begin{table}[h]
\centering
\begin{tabular}{ccc}
\hline
Marks & \begin{tabular}[c]{@{}c@{}}Number of Students\\ Group - A\end{tabular} & \begin{tabular}[c]{@{}c@{}}Number of Students\\ Group - B\end{tabular} \\ \hline
25-30 & 11 & 10 \\ 
30-35 & 18 & 16 \\ 
35-40 & 21 & 22 \\ 
40-45 & 26 & 28 \\ 
45-50 & 14 & 9 \\ \hline
\end{tabular}
\end{table}

  \begin{enumerate}
    \item
	What is data \hfill 1
    \item
	What are the disadvantages of secondary data? \hfill 2
    \item  
	Calculate the arithmetic mean of Group - A \hfill 3
    \item
	Compute the combined mean. Is it greather than the arithmetic mean of Group - B?  \hfill 4 \\Explain the possible reason(s).
\end{enumerate}
  
\end{enumerate}


\end{document}