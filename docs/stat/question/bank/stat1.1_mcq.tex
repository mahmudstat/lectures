\documentclass{exam}
%\documentclass[11pt,a4paper]{exam}
\usepackage{amsmath,amsthm,amsfonts,amssymb,dsfont}
\usepackage{ifthen}
\usepackage{enumerate}% http://ctan.org/pkg/enumerate
\usepackage{multicol}



% Accumulate the answers. Unmodified from Phil Hirschorn's answer
% https://tex.stackexchange.com/questions/15350/showing-solutions-of-the-questions-separately/15353
\newbox\allanswers
\setbox\allanswers=\vbox{}

\newenvironment{answer}
{%
    \global\setbox\allanswers=\vbox\bgroup
    \unvbox\allanswers
}%
{%
    \bigbreak
    \egroup
}

\newcommand{\showallanswers}{\par\unvbox\allanswers}
% End Phil's answer


% Is there a better way?
\newcommand*{\getanswer}[5]{%
    \ifthenelse{\equal{#5}{a}}
    {\begin{answer}\thequestion. (a)~#1\end{answer}}
    {\ifthenelse{\equal{#5}{b}}
        {\begin{answer}\thequestion. (b)~#2\end{answer}}
        {\ifthenelse{\equal{#5}{c}}
            {\begin{answer}\thequestion. (c)~#3\end{answer}}
            {\ifthenelse{\equal{#5}{d}}
                {\begin{answer}\thequestion. (d)~#4\end{answer}}
                {\begin{answer}\textbf{\thequestion. (#5)~Invalid answer choice.}\end{answer}}}}}
}

\setlength\parindent{0pt}
%usage \choice{ }{ }{ }{ }
%(A)(B)(C)(D)
\newcommand{\fourch}[5]{
    \par
    \begin{tabular}{*{4}{@{}p{0.23\textwidth}}}
        (a)~#1 & (b)~#2 & (c)~#3 & (d)~#4
    \end{tabular}
    \getanswer{#1}{#2}{#3}{#4}{#5}
}

%(A)(B)
%(C)(D)
\newcommand{\twoch}[5]{
    \par
    \begin{tabular}{*{2}{@{}p{0.46\textwidth}}}
        (a)~#1 & (b)~#2
    \end{tabular}
    \par
    \begin{tabular}{*{2}{@{}p{0.46\textwidth}}}
        (c)~#3 & (d)~#4
    \end{tabular}
    \getanswer{#1}{#2}{#3}{#4}{#5}
}

%(A)
%(B)
%(C)
%(D)
\newcommand{\onech}[5]{
    \par
    (a)~#1 \par (b)~#2 \par (c)~#3 \par (d)~#4
    \getanswer{#1}{#2}{#3}{#4}{#5}
}

\newlength\widthcha
\newlength\widthchb
\newlength\widthchc
\newlength\widthchd
\newlength\widthch
\newlength\tabmaxwidth

\setlength\tabmaxwidth{0.96\textwidth}
\newlength\fourthtabwidth
\setlength\fourthtabwidth{0.25\textwidth}
\newlength\halftabwidth
\setlength\halftabwidth{0.5\textwidth}

\newcommand{\choice}[5]{%
\settowidth\widthcha{AM.#1}\setlength{\widthch}{\widthcha}%
\settowidth\widthchb{BM.#2}%
\ifdim\widthch<\widthchb\relax\setlength{\widthch}{\widthchb}\fi%
    \settowidth\widthchb{CM.#3}%
\ifdim\widthch<\widthchb\relax\setlength{\widthch}{\widthchb}\fi%
    \settowidth\widthchb{DM.#4}%
\ifdim\widthch<\widthchb\relax\setlength{\widthch}{\widthchb}\fi%

% These if statements were bypassing the \onech option.
% \ifdim\widthch<\fourthtabwidth
%     \fourch{#1}{#2}{#3}{#4}{#5}
% \else\ifdim\widthch<\halftabwidth
% \ifdim\widthch>\fourthtabwidth
%     \twoch{#1}{#2}{#3}{#4}{#5}
% \else
%      \onech{#1}{#2}{#3}{#4}{#5}
%  \fi\fi\fi}

% Allows for the \onech option.
\ifdim\widthch>\halftabwidth
    \onech{#1}{#2}{#3}{#4}{#5}
\else\ifdim\widthch<\halftabwidth
\ifdim\widthch>\fourthtabwidth
    \twoch{#1}{#2}{#3}{#4}{#5}
\else
    \fourch{#1}{#2}{#3}{#4}{#5}
\fi\fi\fi}



\begin{document}
\begin{titlepage}
    \begin{center}
        \vspace*{1cm}
            
        \Huge
        \textbf{Statistics MCQ Question Bank}
            
        \vspace{0.5cm}
        \LARGE
        First Paper \\
        Chapter 01 \\
         \textbf{Statistics, Variable and Concepts of Different Symbols}
            
        \vspace{1.5cm}
            
        \textbf{Abdullah Al Mahmud}
            
        \vfill
            
            
        \vspace{0.8cm}
            
            
        \Large
        www.statmania.info\\
            
    \end{center}
\end{titlepage}


\begin{questions}

\question \textbf{In which scale of measurement, zero is regarded as true zero?}
\choice{Nominal scale}{Interval scale}{Ratio scale}{Ordinal scale}{c}

\question \textbf{Which is a discrete variable?}
\choice{Weight}{Amount of rainfall}{Distance}{Grade in a subject}{d}

\question \textbf{$If x_1=2, x_2=-3, x_3=7$, and $x_4=12, \displaystyle \sum_{i=1}^4 x_i^2=?$}
\choice{26}{106}{206}{216}{c}

\question \textbf{Which one falls in the category of interval scale?}
\choice {Temperature}{Speed}{Distance}{Film rating}{a}

\question \textbf{Which one is product of square?}
\choice {$\prod x_i^2$}{$(\prod x_i)^2$}{$\sum x_i^2 \times \sum x$}{$\sum x_i^2$}{a}

\question \textbf{For which variable, determining number of terms is not possible?}
\choice{Discrete variable}{Continuous variable}{Quantitative variable}{Qualitative variable}{b}

\textbf{Answer the next three question based on the following information.}

\textbf{A farmer collects growth (in cm) of 10 plants in a month and finds that \\ $\sum x_i = 7$ and $\sum x_i^2=15$}

\question \textbf{What is the value of $\sum (x_i+4)$?}
\choice{23}{$\sum x_i +4n$}{22}{11}{a}

\question \textbf{What is the value of $\sum (x_i-4)^2$?}
\choice{23}{135}{484}{121}{a}

\question \textbf{If the square of summation is subtracted the sum of square, the value is - }
\choice{-8}{34}{8}{-34}{d}

\question \textbf{Which one is not an example of ratio scale?}
\choice{Room no.}{Income}{Number of accidents}{Weight}{a}


%\question \textbf{To complete the song, the last answer should be
%\choice{a}{b}{c}{d}{e} % Invalid answer choice

\end{questions}

%\newpage  %Uncomment to put on new age
%\bigskip

\begin{multicols}{3}
[
Answer Key: 
]
\showallanswers % Phil Hirschorn
\end{multicols}


\end{document}