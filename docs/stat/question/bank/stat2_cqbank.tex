\documentclass[a4paper,oneside, margin=1.4in]{book}

\usepackage{lipsum}
\usepackage[hidelinks]{hyperref}
\usepackage{titletoc}
\usepackage{amsmath}
\usepackage{geometry}
\usepackage{graphicx}
\graphicspath{ {.} }
\geometry{a4paper, margin=1in}
\titlecontents*{chapter}
  [0pt]% <left>
  {}
  {\chaptername\ \thecontentslabel\quad}
  {}
  {\bfseries\hfill\contentspage}    

\usepackage{bookmark}
\usepackage{etoolbox}

\makeatletter
\newcommand*{\AddChapterPrefixInBookmarks}{%
  \if@mainmatter
    \ifnum\bookmarkget{level}=0 %
      \preto\bookmark@text{\@chapapp\space}%
    \fi
  \fi
}
\makeatother

\bookmarksetup{
  numbered,
  addtohook=\AddChapterPrefixInBookmarks,
}

% Workaround for numbered sections in unnumbered
% chapter "Introduction" to avoid chapter number
% zero.
\renewcommand*{\thesection}{%
  \ifcase\value{chapter}%
  \else
    \thechapter.%
  \fi
  \arabic{section}%
}

\title{My document}

\begin{document}
\frontmatter

\begin{titlepage}
    \begin{center}
        \vspace*{1cm}
            
        \Huge
        \textbf{Statistics Question Bank}
            
        \vspace{0.5cm}
        \huge
        Second Paper
            
        \vspace{1.5cm}
            
        \textbf{Abdullah Al Mahmud}

     \vspace{1.5cm}

	\Large 
	Updated on: \today
            
        \vfill
            

            
        \vspace{0.8cm}
            
\includegraphics[width=1cm]{logo}
            
        \Large
        www.statmania.info\\
            
    \end{center}
\end{titlepage}


\tableofcontents


\mainmatter
\chapter{Probability} 
\section{Creative Questions}

\begin{enumerate}
  \item
  \textbf{It is observed that in a college, there are 100 students, of whom 30 play football, 40 play cricket, and 20 play both.}
 
  \begin{enumerate}
    \item
	What is the range of probability? \hfill 1
    \item
    	What is the relationship between independence and mutual excluvity?  \hfill 2
    \item
    	Are the probabilities of playing cricket and that of football independent? Prove. \hfill 3
     \item
     	If a student is selected randomly, and if he does not play cricket, what is the probability that \\ he plays football? \hfill 4
  \end{enumerate}


  \end{enumerate}

\section{Short Questions}

  \begin{enumerate}
    \item
	Question \hfill 1
    \item
	Question \hfill 2
    \item  
	Question \hfill 3
    \item
	Question \hfill 4
  \end{enumerate}

\chapter{Random Variable and Probability Function} 
\section{Creative Questions}

  \begin{enumerate}
 \item
  \textbf{The probability density function of a continuous random variable is}

$$
  f(x) =
\begin{cases}
kx^2+kx+ \frac 18,  & 0 \le x \le 2 \\
0, & otherwise
\end{cases}
$$

  \begin{enumerate}
    \item
	What is a continuous random variable? \hfill 1
    \item
    	Find the value of k \hfill 2
    \item
    	Find the probability that the values of x would lie between1 and 3. \hfill 3
     \item
     	Find the 40th percentile of the distribution and explain.  \hfill 4
  \end{enumerate}
  \end{enumerate}

\section{Short Questions}

  \begin{enumerate}
    \item
		What is a continuous random variable?  \hfill 1
    \item
	Question \hfill 1
    \item  
	Question \hfill 1
    \item
	Question \hfill 1
  \end{enumerate}

\chapter{Mathematical Expectation} 
\section{Creative Questions}


\section{Short Questions}

\chapter{Binomial Distribution} 
\section{Creative Questions}


\section{Short Questions}

\chapter{Poisson Distribution} 
\section{Creative Questions}

  \begin{enumerate}
  \item
  \textbf{In winter, the probability that it rains on a particular day is 0.015. An analyst observes \\ 100 winter days.}
 
  \begin{enumerate}
    \item
	What is an experiment? \hfill 1
    \item
    	When can the Poisson distribution be approximated by the Binomial distribution? \hfill 2
    \item
    	Find, using Binomial distribution, the probability that  it would not rain at all on the \\ observed days. \hfill 3
     \item
     	Find the probability in 3(c) using Poisson distribution.  \hfill 4
  \end{enumerate}

  \end{enumerate}

\section{Short Questions}

\chapter{Normal Distribution} 
\section{Creative Questions}


\section{Short Questions}

\chapter{Index Number} 
\section{Creative Questions}


\section{Short Questions}

\chapter{Sampling} 
\section{Creative Questions}


\section{Short Questions}

\chapter{Vital Statistics} 
\section{Creative Questions}


  \begin{enumerate}
 \item
  \textbf{For projection of population in a future time period, demographers use simple, \\ geometric or exponential growth technique. Each method has its advantages and \\ disadvantages.}

  \begin{enumerate}
    \item
	What is geometric growth? \hfill 1
    \item
    	In geometric growth method, obtain the formula for time required for the population to get \\ doubled [denote rate as r]. \hfill 2
    \item
    	In exponential method, how much unit of time is required for the population to get tripled?  \hfill 3
     \item
     	For projecting (predicting future values), is geometric growth method better than the \\ exponential method? Justify.  \hfill 4
  \end{enumerate}
  \end{enumerate}

\section{Short Questions}

\backmatter
\chapter{Conclusion}
\lipsum[8]

\tableofcontents
\end{document}