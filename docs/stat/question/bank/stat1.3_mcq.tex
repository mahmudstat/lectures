\documentclass{exam}
%\documentclass[11pt,a4paper]{exam}
\usepackage{amsmath,amsthm,amsfonts,amssymb,dsfont}
\usepackage{ifthen}
\usepackage{enumerate}% http://ctan.org/pkg/enumerate
\usepackage{multicol}
\usepackage{graphicx}



% Accumulate the answers. Unmodified from Phil Hirschorn's answer
% https://tex.stackexchange.com/questions/15350/showing-solutions-of-the-questions-separately/15353
\newbox\allanswers
\setbox\allanswers=\vbox{}

\newenvironment{answer}
{%
    \global\setbox\allanswers=\vbox\bgroup
    \unvbox\allanswers
}%
{%
    \bigbreak
    \egroup
}

\newcommand{\showallanswers}{\par\unvbox\allanswers}
% End Phil's answer


% Is there a better way?
\newcommand*{\getanswer}[5]{%
    \ifthenelse{\equal{#5}{a}}
    {\begin{answer}\thequestion. (a)~#1\end{answer}}
    {\ifthenelse{\equal{#5}{b}}
        {\begin{answer}\thequestion. (b)~#2\end{answer}}
        {\ifthenelse{\equal{#5}{c}}
            {\begin{answer}\thequestion. (c)~#3\end{answer}}
            {\ifthenelse{\equal{#5}{d}}
                {\begin{answer}\thequestion. (d)~#4\end{answer}}
                {\begin{answer}\textbf{\thequestion. (#5)~Invalid answer choice.}\end{answer}}}}}
}

\setlength\parindent{0pt}
%usage \choice{ }{ }{ }{ }
%(A)(B)(C)(D)
\newcommand{\fourch}[5]{
    \par
    \begin{tabular}{*{4}{@{}p{0.23\textwidth}}}
        (a)~#1 & (b)~#2 & (c)~#3 & (d)~#4
    \end{tabular}
    \getanswer{#1}{#2}{#3}{#4}{#5}
}

%(A)(B)
%(C)(D)
\newcommand{\twoch}[5]{
    \par
    \begin{tabular}{*{2}{@{}p{0.46\textwidth}}}
        (a)~#1 & (b)~#2
    \end{tabular}
    \par
    \begin{tabular}{*{2}{@{}p{0.46\textwidth}}}
        (c)~#3 & (d)~#4
    \end{tabular}
    \getanswer{#1}{#2}{#3}{#4}{#5}
}

%(A)
%(B)
%(C)
%(D)
\newcommand{\onech}[5]{
    \par
    (a)~#1 \par (b)~#2 \par (c)~#3 \par (d)~#4
    \getanswer{#1}{#2}{#3}{#4}{#5}
}

\newlength\widthcha
\newlength\widthchb
\newlength\widthchc
\newlength\widthchd
\newlength\widthch
\newlength\tabmaxwidth

\setlength\tabmaxwidth{0.96\textwidth}
\newlength\fourthtabwidth
\setlength\fourthtabwidth{0.25\textwidth}
\newlength\halftabwidth
\setlength\halftabwidth{0.5\textwidth}

\newcommand{\choice}[5]{%
\settowidth\widthcha{AM.#1}\setlength{\widthch}{\widthcha}%
\settowidth\widthchb{BM.#2}%
\ifdim\widthch<\widthchb\relax\setlength{\widthch}{\widthchb}\fi%
    \settowidth\widthchb{CM.#3}%
\ifdim\widthch<\widthchb\relax\setlength{\widthch}{\widthchb}\fi%
    \settowidth\widthchb{DM.#4}%
\ifdim\widthch<\widthchb\relax\setlength{\widthch}{\widthchb}\fi%

% These if statements were bypassing the \onech option.
% \ifdim\widthch<\fourthtabwidth
%     \fourch{#1}{#2}{#3}{#4}{#5}
% \else\ifdim\widthch<\halftabwidth
% \ifdim\widthch>\fourthtabwidth
%     \twoch{#1}{#2}{#3}{#4}{#5}
% \else
%      \onech{#1}{#2}{#3}{#4}{#5}
%  \fi\fi\fi}

% Allows for the \onech option.
\ifdim\widthch>\halftabwidth
    \onech{#1}{#2}{#3}{#4}{#5}
\else\ifdim\widthch<\halftabwidth
\ifdim\widthch>\fourthtabwidth
    \twoch{#1}{#2}{#3}{#4}{#5}
\else
    \fourch{#1}{#2}{#3}{#4}{#5}
\fi\fi\fi}



\begin{document}
\begin{titlepage}
    \begin{center}
        \vspace*{1cm}
            
        \Huge
        \textbf{Statistics MCQ Question Bank}
            
        \vspace{0.5cm}
        \LARGE
        First Paper \\
                \vspace*{2cm}
        Chapter 03 \\

         \textbf{Measures of Central Tendency} \\
         

            
            
        \vfill
            
            
        \vspace{0.8cm}
            
                    \textbf{Abdullah Al Mahmud}
                            \vspace{0.8cm}
                    
                            \includegraphics[width=1cm]{logo}
                            
        \Large
        www.statmania.info\\
                 	Updated on: \today

            
    \end{center}
\end{titlepage}


\begin{questions}

\question \textbf{The arithmetic mean of first n natural numbers-}
\choice {$\frac{n}{2}$}{$\frac{n+1}{2}$}{$\frac{n^2}{2}$}{$\frac{n^2-1}{2}$}{b}

\question \textbf{When is the relationship $AM = HM = GM$ true?}
\choice {All values are equal}{The values form a geometric progression}{ The values form an arithmetic progression}{All values are distinct}{a}

\question \textbf{In the presence of outlier(s), which measure of central tendency is suitable?}
\choice {Arithmetic mean}{Median}{Quadratic mean}{Power mean}{b}

\question \textbf{If a rate is defined as $R = \frac cd$, where c is constant, then which measure is perfect?}
\choice {Weighted arithmetic mean}{Harmonic mean}{Quadratic mean}{Weighted geometric mean}{b}

\textbf{Answer the next two questions as per the following information.}

42 44 59 64 70 72 74 91 94 are 9 values.

\question \textbf{What is the 50th percentile?}
\choice {64}{70}{72}{71}{b}

\question \textbf{Below which value lie 70 percent values?}
\choice {42}{44}{59}{74}{d}

\question \textbf{Which measure might have more than one value?}
\choice{Arithmetic mean}{Geometric mean}{Quadratic mean}{Mode}{d}

\question \textbf{Above which value lie 30\% observations?}
\choice{3rd Quartile}{Median}{30th Percentile}{70th percentile}{d}

\question \textbf{Arithmetic means of three groups having equal no. of items are 30, 32, and 34. What is the combined mean?}
\choice{30.33}{32.67}{32.00}{33.00}{c}

%\question \textbf{To complete the song, the last answer should be
%\choice{a}{b}{c}{d}{e} % Invalid answer choice

\end{questions}

\newpage  %Uncomment to put on new age
\bigskip

\begin{multicols}{3}
[
Answer Key: 
]
\showallanswers % Phil Hirschorn
\end{multicols}


\end{document}