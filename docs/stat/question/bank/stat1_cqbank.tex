\documentclass[a4paper,oneside]{book}

\usepackage{lipsum}
\usepackage[hidelinks]{hyperref}
\usepackage{titletoc}
\titlecontents*{chapter}
  [0pt]% <left>
  {}
  {\chaptername\ \thecontentslabel\quad}
  {}
  {\bfseries\hfill\contentspage}    

\usepackage{bookmark}
\usepackage{etoolbox}

\makeatletter
\newcommand*{\AddChapterPrefixInBookmarks}{%
  \if@mainmatter
    \ifnum\bookmarkget{level}=0 %
      \preto\bookmark@text{\@chapapp\space}%
    \fi
  \fi
}
\makeatother

\bookmarksetup{
  numbered,
  addtohook=\AddChapterPrefixInBookmarks,
}

% Workaround for numbered sections in unnumbered
% chapter "Introduction" to avoid chapter number
% zero.
\renewcommand*{\thesection}{%
  \ifcase\value{chapter}%
  \else
    \thechapter.%
  \fi
  \arabic{section}%
}

\title{My document}
\pagestyle{headings}

\begin{document}
\frontmatter

\begin{titlepage}
    \begin{center}
        \vspace*{1cm}
            
        \Huge
        \textbf{Statistics Question Bank}
            
        \vspace{0.5cm}
        \LARGE
        First Paper
            
        \vspace{1.5cm}
            
        \textbf{Abdullah Al Mahmud}
            
        \vfill
            
            
        \vspace{0.8cm}
            
            
        \Large
        www.statmania.info\\
            
    \end{center}
\end{titlepage}

\tableofcontents


\mainmatter

\chapter{Statistics, Variable and Concepts of Different Symbols} 


\section{Creative Questions}

\begin{enumerate}
  \item
  \textbf{Income and expenditure (both in thousands) of some individuals are collected:}
 

\begin{table}[h]
 \begin{center}
\begin{tabular}{l|l|l|l|l}

Income (x)  & 20 & 30 & 25 & 10 \\ \hline
Expenditure (y) & 15  & 27  & 18 & 5 \\ 
\end{tabular}
\end{center}
\end{table}


  \begin{enumerate}
    \item
	What is a discrete variable? \hfill 1
    \item
    	Can fractional numbers be discrete? Explain briefly.  \hfill 2
    \item
    	Are, in the stem, $\displaystyle \sum_{i=1}^{n} \sum_{i=1}^{n} x_iy_j = \sum_{i=1}^{n} x_iy_i?$ Show statistically. \hfill 3
     \item
     	Prove empirically that sum of square is unequal to square of sum of numbers. \hfill 4
  \end{enumerate}
  
   \item
  \textbf{Call duration of 6 calls in a customer care center are}
  
  \begin{center}
  2, 2.5, 1.5, 5, 6, 3
  \end{center}
 
  \begin{enumerate}
    \item
	What is a sample? \hfill 1
    \item
    	Are all quantitative variables continuous?  \hfill 2
    \item
    	Determine $\displaystyle \sum_{i=1}^7 (x_i-3)^3$ \hfill 3
     \item
     	Find the values of  $\displaystyle \sum_{i=1}^7 (x_i-5)^2$ and $\displaystyle \sum_{i=1}^7 x_i^2+5.$  \hfill 4 \\
     	Explain mathematically why they are unequal.
  \end{enumerate}
  
   \item
	  \textbf{Goals scored by Karim Benzema in five seasons are recorded to be the following:} 
	  
	  \begin{table}[h]
	  \centering
\begin{tabular}{c|c|c}
Season & La Liga (x) & Uefa Champions League (y) \\ \hline
2017-18 & 5 & 5 \\ 
2018-19 & 21 & 4 \\
2019-20 & 21 & 5 \\
2020-21 & 23 & 6 \\ 
2021-22 & 27 & 15 \\ \hline
\end{tabular}
\end{table}
  
  \begin{enumerate}
    \item
	What is a quantitative variable? \hfill 1
    \item
	What is the notation to denote his total number of goals? \hfill 2
    \item  
	Compute $\displaystyle \sum_{i=1}^5 (y_i - 3)^2$ \hfill 
    \item
	Find total number of goals using two different notations and examine whether they match. \hfill 4
  \end{enumerate}

  
 \item
	  \textbf{Below are some information} 
  
  $x_1=3, x_2=4, x_3=1, x_4=0 \\
	  y_1=1, y_2=5, y_3=0, y_4=2$
  
  \begin{enumerate}
    \item
	What is a qualitative variable? \hfill 1
    \item
	Find $\displaystyle \sum_{i=1}^{4}x_i^2$ \hfill 2
    \item  
	Prove that $\displaystyle \sum_{i=1}^{4} (x_i+y_i) = \sum_{i=1}^{4}x_i + \sum_{i=1}^{4}y_i $ \hfill 3
    \item
	Find the value of $\displaystyle \sum_{i=1}^{4} x_iy_i-\sum_{i=1}^{4} x_i+4$ \hfill 4

  \end{enumerate}
  
  \end{enumerate}

\section{Short Questions}

\chapter{Data Collection, Presentation, and Organization of Data} 

\section{Creative Questions}
\begin{enumerate}

    \item
  \textbf{Frequency distribution of marks in statistics of a college is given in the following table.}
 

\begin{table}[h]
\centering
\begin{tabular}{ccc}
\hline
Marks & \begin{tabular}[c]{@{}c@{}}Number of Students\\ Group - A\end{tabular} & \begin{tabular}[c]{@{}c@{}}Number of Students\\ Group - B\end{tabular} \\ \hline
25-30 & 11 & 10 \\ 
30-35 & 18 & 16 \\ 
35-40 & 21 & 22 \\ 
40-45 & 26 & 28 \\ 
45-50 & 14 & 9 \\ \hline
\end{tabular}
\end{table}

  \begin{enumerate}
    \item
	What is data \hfill 1
    \item
	What are the disadvantages of secondary data? \hfill 2
    \item  
	Calculate the arithmetic mean of Group - A \hfill 3
    \item
	Compute the combined mean. Is it greather than the arithmetic mean of Group - B? Explain the possible reason(s). \hfill 4
\end{enumerate}
 \end{enumerate}
  
  
\section{Short Questions}

\chapter{Measures of Central Tendency} 
\section{Creative Questions}
\begin{enumerate}
    \item
  \textbf{In the test examination, marks of 11 students in statistics are: 90, 92, 93, 49, 44, 88, 80, 58, 83, 71, 76.}
  \begin{enumerate}
    \item
	What is central tendency? \hfill 1
    \item
	When is median better than arithmetic mean? Explain with an example. \hfill 2
    \item  
	Find the 3rd the quartile and 61st percentile from the data and explain.  \hfill 3
    \item
	Do quantiles depend on change of origin and scale. Prove using two examples.\hfill 4
\end{enumerate}



      \item
  \textbf{Scores of a batsman in the last 20 innings are} 
   \begin{center}
  	 28, 30, 16, 48, 50, 86, 105, 20, 10, 36, \\
  	 12, 25, 20, 35, 65, 12, 10, 76, 55, 32
  	 \end{center}
  \begin{enumerate}
    \item
	Write down the formula of weighted harmonic mean \hfill 1
    \item
	Can median be a better measure of central tendency than arithmetic mean for this data?  \hfill 2
    \item  
	Draw a stem and leaf plot from the data and explain.  \hfill 3
    \item
	Make a frequency distribution from the data and also find and interpret cumulative  \hfill 4 \\ frequencies and percentages.
\end{enumerate}

 \item
	  \textbf{In ODI cricket, two top batsmen are (as of 2nd Sept, 2022) Babar Azam and Rassie van der Dussen. Their average (arithmetic mean) scores are 59.79 and 69.32, appearing in 90 (including being not out in 12 occassions) and 33 (including being not out in 11 occassions) matches, respectively.} 
  
  \begin{enumerate}
    \item
	When is arithmetic mean inappropriate to use? \hfill 1
    \item
	Is arithmetic mean always suitable for comparison? \hfill 2
    \item  
	Find the combined arithmetic mean and explain. \hfill 3
    \item
	How to compare two sets of data having significantly distinct ranges? \hfill 4
  \end{enumerate}
  
   \item
	  \textbf{A fridge manufacturing company observe temperatures of newly developed 8 deep fridges. The observed temperatures (in degree celsius are:} 
	  
	    \begin{center}
-10, -8, -2, -4, -4, -1, -12, -3, -13
  \end{center}
  
  \begin{enumerate}
    \item
	What is a Decile? \hfill 1
    \item
	How many Deciles does a data set have? Why? \hfill 2
    \item  
	Compute the 8th Decile from the data and explain. \hfill 3
    \item
	Find and compare arithmetic and geometric mean from the data. \hfill 4
  \end{enumerate}
  
   \item
	  \textbf{Given below is a series of data.} 
	  
	  	    \begin{center}
	 $5, 7, 9, \cdots , 123$
	    \end{center}
  
  \begin{enumerate}
    \item
	What is the summation of natural numbers up to nth value? \hfill 1
    \item
	Find the arithmetic mean of natural numbers from 1 up to 20. \hfill 2
    \item  
	Find the arithmetic mean of the given series. \hfill 3
    \item
	Prove that arithmetic mean is greater than gemetric mean theoretically and empricially. \hfill 4
  \end{enumerate}
  
   \item
	  \textbf{Grades of a an undergraduate student with major in statistics are given} 

\begin{table}[h]
\centering
\begin{tabular}{c|c|c}
\hline
Course & Grade & Credit \\ \hline
Probability & 3.75 & 4 \\ 
Simulation & 3.50 & 3 \\ 
Calculas & 3.50 & 4 \\ 
Linear Algebra & 3.75 & 4 \\ 
Econometrics & 3.00 & 2 \\ 
Programming & 3.50 & 3 \\ \hline
\end{tabular}
\end{table}

  
  \begin{enumerate}
    \item
	Write down the formula of weighted mean. \hfill 1
    \item
	What is difference between weight and frequency? \hfill 2
    \item  
	Determine the GPA of the student. \hfill 3
    \item
	Determine the geometric mean for the data and evaluate \\ suitability. \hfill 4
  \end{enumerate}

 \item
	  \textbf{A student walks 3 hours at 5 km per hour (kph), 4 hours at 4 kph, and 2 hours at 3 kph} 
  
  \begin{enumerate}
    \item
	When is harmonic mean suitable? \hfill 1
    \item
	Which means could we use for the given data and why? \hfill 2
    \item  
	Find the average speed using weighted harmonic mean. \hfill 3
    \item
	Find the average speed using another method and mathematically show their relationship. \hfill 4
  \end{enumerate}

 \item
	  \textbf{A clyclist moves around a square-shaped lake with the speeds 20, 25, 30, and 16 km per hour.} 
  
  \begin{enumerate}
    \item
	What is grouped data? \hfill 1
    \item
	Is arithmetic mean suitable for this data? \hfill 2
    \item  
	Find the average speed of the cyclist. \hfill 3
    \item
	Can we use some other formula for finding the average? Demonstrate. \hfill 4
  \end{enumerate}


\end{enumerate}

\section{Short Questions}

\chapter{Measures of Dispersion} 
\section{Creative Questions}

\begin{enumerate}
    \item
  \textbf{Temperatures of two cold regions for five days are as below:}

    City A: 2, 1, -1, 0, 3

    City B: 3, 0, -2, 2, 3
  \begin{enumerate}
    \item
	What is standard deviation?? \hfill 1
    \item
	Is standard deviation of a set of negative values negative? Justify mathematically. \hfill 2
    \item  
	Find Mean Deviation about mean of the values of city A.  \hfill 3
    \item
	Which city has more consistent weather? Verify statistically. \hfill 4
\end{enumerate}


\end{enumerate}

\section{Short Questions}

\chapter{Moments, SKewness, and Kurtosis} 
\section{Creative Questions}

  \begin{enumerate}
  
   \item
	  \textbf{The arithmetic and geometric means of the first and third quartiles of a distribution are 10 and 8, respectively. The second quartile is 10.} 
  
  \begin{enumerate}
    \item
	What is the formula suggested by Pearson to find skewness? \hfill 1
    \item
	Which moments are useful in measuring central tendency and dispersion?  \hfill 2
    \item  
	Find skewness from the stem using a suitable formula. \hfill 3
    \item
	Which method of finding skewness od you think is the best and why? \hfill 4
\end{enumerate}
 \item
	  \textbf{For a particular data set, Median = 120, Mode = 110, Standard Deviation = 4, and Coefficient of Variation (CV)  = 3.2} 
  
  \begin{enumerate}
    \item
	Why is  CV used?  \hfill 1
    \item
	Find arithmetic mean.. \hfill 2
    \item  
	Find skewness according to Pearson's method ($SK_P$) \hfill 3
    \item
	Does ($SK_P$) convey the proper idea about the data as to the given information? Justify. \hfill 4
  \end{enumerate}
  \end{enumerate}

  \begin{enumerate}
 \item
	  \textbf{US Dollar exchange (to taka) in Bangladesh since 1980 to 2005 (after each 5 years) were: \\ 16, 31, 36, 40, 52, 64} 
  
  \begin{enumerate}
    \item
	What are moments? \hfill 1
    \item
	Which moment is equal to the variance? Show mathematically. \hfill 2
    \item  
	Find, from the stem, the first and second raw moments about 1. \hfill 3
    \item
	Find skewness and kurtosis of and explain. \hfill 4
\end{enumerate}

 \item
	  \textbf{The first four moments about 3 of a distribution are -1, 5, -10, and 120.} 
  
  \begin{enumerate}
    \item
	What are moments used for? \hfill 1
    \item
	Can the second central moment be greater than the third central moment? \hfill 2
    \item  
	Find the second and third moments about arithmetic mean of the distribution. \hfill 3
    \item
	Find skewness and kurtosis and comment on the values.  \hfill 4
\end{enumerate}

 \item
	  \textbf{Marks obtained by a student in 7 subjects are} 
	  \begin{center}
	  70, 66, 55, 45, 80, 30, 82
	\end{center}
  
  \begin{enumerate}
    \item
	What is negative skewness? \hfill 1
    \item
	Draw graphs of positive and negative skewness showing the locations of mean and median. \hfill 2
    \item  
	Determine the five number summary from the stem and explain. \hfill 3
    \item
	Are the data symmetric? If not, comment on the pattern of data. \hfill 4
\end{enumerate}

 \item
	  \textbf{United Nations Children's Fund (UNICEF) is an agency of the United Nations responsible for providing humanitarian and developmental aid to children worldwide. A  UNICEF researcher collected heights of 7 children for a project, and the heights are} 

	\begin{center}
	  2.2, 2.15, 1.9, 3.1, 2.7, 3.0, 3.5
	  	\end{center}
  
  \begin{enumerate}
    \item
	Which value are central moments estimated around? \hfill 1
    \item
	Moments around origin (0) are central moments - Comment. \hfill 2
    \item  
	Find the first central moment. \hfill 3
    \item
	Find the skewness of the data and interpret.  \hfill 4
  \end{enumerate}
  
   \item
	  \textbf{A researcher wants to compare average life time of people in Bangladesh and other countries. He collected life time of 10 people in Bangladesh.} 
	  
	  	\begin{center}
	  75, 62, 63, 72, 66, 76, 59, 77, 70, 79
	  	\end{center}
  
  \begin{enumerate}
    \item
	What is symmetry? \hfill 1
    \item
	Mathematically show the theoretical value of the first central moment. \hfill 2
    \item  
	Compute the 2nd, 3rd, and 4th central moments of the data. \hfill 3
    \item
	Estimate skewness and kurtosis and explain. \hfill 4
  \end{enumerate}

\end{enumerate}

\section{Short Questions}

\chapter{Correlation and Regression} 
\section{Creative Questions}
\section{Short Questions}

\chapter{Time Series} 
\section{Creative Questions}
  \begin{enumerate}
 \item
	  \textbf{GDP (in bn. US\$ PPP) of Bangladesh since 1980 to 1985 according to an estimate \\ of International  Monetary Fund: 41.2, 47.4, 52.0, 56.5, 61.0, 65.3}
  \begin{enumerate}
    \item
	What is time series data? \hfill 1
    \item
	What are the components of a time series model? \hfill 2
    \item  
	Determine the 3-yearly moving average from the data. \hfill 3
    \item
	Find trend of the data using another method (other than (c)), plot both, and comment \\ which is better. \hfill 4
\end{enumerate}

 \item
	  \textbf{Annual sales of company are as given in the following}\
	  
	  \begin{table}[h]
	  \centering
\begin{tabular}{l|l|l|l|l|l|l|l}
Year & 2010 & 2011 & 2012 & 2013 & 2014 & 2015 & 2016 \\ \hline
Profit (million) & 40 & 45 & 46 & 53 & 65 & 70 & 73
\end{tabular}
\end{table}

  \begin{enumerate}
    \item
	What is a trend? \hfill 1
    \item
	Do the data in the stem seem to have a trend? \hfill 2
    \item  
	Find the trend using semi-average method. \hfill 3
    \item
	Find the trend using 2-yearly moving average method. Would it better if we used 3-yearly  \hfill 4 \\  method?
\end{enumerate}

 \item
	  \textbf{Income of a freelancer in 6 successive months (from Jan to Jun) was found to be \\ 46.0, 49.5, 51.5, 50.6, 56.5, and 60 (in thousands BDT.).}
  \begin{enumerate}
    \item
	What is time series data? \hfill 1
    \item
	What are the components of a time series model? \hfill 2
    \item  
	Determine the 3-monthly moving average from the data. \hfill 3
    \item
	Draw the moving averages on a graph paper and interpret. \hfill 4
\end{enumerate}

\end{enumerate}

\section{Short Questions}

\chapter{Published Statistics in Bangladesh} 
\section{Creative Questions}
\section{Short Questions}

\backmatter
\chapter{Conclusion}
\lipsum[8]

\tableofcontents
\end{document}